\chapter{The Monophoton Analysis}
\label{chap:analysis}

The main event.

\section{Event Selection}
\label{sec:event_selection}

\section{Irreducible backgrounds}
\label{sec:irreducible}

\begin{figure}[htbp]
  \centering
    \includegraphics[width=0.49\textwidth]{Analysis/Figures/RZee.pdf}
    \includegraphics[width=0.49\textwidth]{Analysis/Figures/RZmm.pdf}
    \caption{
      Transfer factors \RZee\ (left) and \RZmm\ (right). 
      The uncertainty bands in green (inner) and orange (outer) show the systematic uncertainty, and the combination of systematic and statistical uncertainty arising from limited MC sample size, respectively. 
      The systematic uncertainties considered are the uncertainties in the data-to-simulation correction factors $\rho$ for the lepton identification efficiencies.
    }
    \label{fig:tf_z}
\end{figure}

Using the transfer factor \RZll, the total estimated event yield \Tll\ in each dilepton control region in the $i^\mathrm{th}$ bin of the \ETg\ distribution can be expressed as
\begin{equation}
  \Tll[,i] = \frac{\NZg[i]}{\RZll[,i]} + b_{\ell\ell\Pgg,i},
\end{equation}
where \NZg\ is the number of \zinvg\ events in the combined signal regions and $b_{\ell\ell\Pgg}$ is the predicted contribution from other background sources in the dilepton control region, namely \ttg, VV\Pgg, and misidentified hadrons. 
The subscript $i$ indicates that the quantities are evaluated in bin $i$ of the \ETg\ distribution.

\begin{figure}[htbp]
  \centering
    \includegraphics[width=0.49\textwidth]{Analysis/Figures/RWe.pdf}
    \includegraphics[width=0.49\textwidth]{Analysis/Figures/RWm.pdf}
    \caption{
      Transfer factors \RWe\ (left) and \RWm\ (right). 
      The uncertainty bands in green (inner) and orange (outer) show the systematic uncertainty, and the combination of systematic and statistical uncertainty arising from limited MC sample size, respectively. 
      The systematic uncertainties considered are the uncertainties in the data-to-simulation correction factors $\rho$ for the lepton identification efficiencies.
    }
    \label{fig:tf_w}
\end{figure}

\begin{figure}[htbp]
  \centering
    \includegraphics[width=0.49\textwidth]{Analysis/Figures/fZW.pdf}
    \caption{
      Transfer factor \fZW. 
      The uncertainty bands in green (inner) and orange (outer) show the systematic uncertainty, and the combination of systematic and statistical uncertainty arising from limited MC sample size, respectively. 
      The systematic uncertainties considered are the uncertainties from higher-order theoretical corrections.
    }
    \label{fig:tf_wz}
\end{figure}

Using \RWl\ and \fZW, the total estimated event yield \Tl\ in each single-lepton control region in the $i^\mathrm{th}$ bin of the \ETg\ distribution can be expressed as
\begin{equation}
  \Tl[,i] = \frac{\NZg[i]}{\RWl[,i]\fZW[,i]} + b_{\ell\Pgg,i},
\end{equation}
where $b_{\ell\Pgg}$ is the predicted contribution from other background sources in the single-lepton regions, namely misidentified electrons and hadrons and other minor SM processes.

\subsection{Higher-order corrections to \vg\ differential cross sections}
\label{sec:theory_uncertainties}

\begin{figure}[htbp]
  \centering
  \includegraphics[width=0.48\linewidth]{Analysis/Figures/ewkcorr_zg.pdf} \\
  \includegraphics[width=0.48\linewidth]{Analysis/Figures/ewkcorr_wgplus.pdf}
  \includegraphics[width=0.48\linewidth]{Analysis/Figures/ewkcorr_wgminus.pdf}
  \caption{
    Electroweak NLO cross section corrections as a function of photon \pt\ for \zinvg\ (top), $\PW^{+}+\Pgg$ (bottom left), and $\PW^{-}+\Pgg$ (bottom right) processes, overlaid with uncertainty bands. 
    See text for descriptions of the individual components of the uncertainty.
    The uncertainty due to \Pgg-induced production is negligible in \zinvg\ production.
  }
  \label{fig:ewk_correction}
\end{figure}

We apply the correction factors shown in Fig.~\ref{fig:ewk_correction}, which are combinations of Sudakov suppression factors and photon-induced enhancements, and are provided by the authors of Ref.~\cite{Denner:2015fca} in addition to the NNLO QCD correction.

\begin{figure}[htbp]
  \centering
    \includegraphics[width=0.45\textwidth]{Analysis/Figures/tf_syst_zg.pdf}
    \includegraphics[width=0.45\textwidth]{Analysis/Figures/tf_syst_wg.pdf}
    \caption{
      Systematic uncertainty in the transfer factors for \zinvg\ (left) and \wlng\ (right). The last bin includes all events with $\ETg > 1000\GeV$.
    }
    \label{fig:tf_syst}
\end{figure}

Figure~\ref{fig:tf_syst} shows the effect of systematic uncertainty in the ratio between the \zinvg\ and \wlng\ processes with respect to nominal value for ${\PZ\Pgg}$ and ${\PW\Pgg}$ respectively.

\section{Misidentified electrons}
\label{sec:efake}

Figure~\ref{fig:efake_fits} shows the six fits performed on \Pe\Pe\ and \Pe\Pgg\ in bins of probe $\pt$, from which the $R_{\Pe}$ factor used for the estimation of the electron misidentification background is derived. 
The $R_{\Pe}$ factor is computed as the ratio of the integral of the signal template function between 81\GeV and 101\GeV.

%%% these fils have charged PF veto included, need to swap out
\begin{figure}[htbp]
  \begin{center}
    \includegraphics[width=0.48\linewidth]{Analysis/Figures/fit_data_ee_pt_175_200.pdf}
    \includegraphics[width=0.48\linewidth]{Analysis/Figures/fit_data_eg_pt_175_200.pdf}
    \includegraphics[width=0.48\linewidth]{Analysis/Figures/fit_data_ee_pt_200_250.pdf}
    \includegraphics[width=0.48\linewidth]{Analysis/Figures/fit_data_eg_pt_200_250.pdf}
    \includegraphics[width=0.48\linewidth]{Analysis/Figures/fit_data_ee_pt_250_6500.pdf}
    \includegraphics[width=0.48\linewidth]{Analysis/Figures/fit_data_eg_pt_250_6500.pdf}
    \caption{
      Fits to the mass distributions for \Pe\Pe\ (left) and \Pe\Pgg\ (right) selections, in bins of probe $\pt$: $175 < \pt < 200\GeV$ (top), $200 < \pt < 250\GeV$ (middle), $\pt > 250\GeV$ (bottom). 
      The blue solid line represents the full fit model, and the green dashed line its background component.
    }
    \label{fig:efake_fits}
  \end{center}
\end{figure}

The proxy sample for the background estimation is obtained by identical event selection as that described in Sec.~\ref{subsec:selection}, but with the pixel-seed veto inverted on the photon candidate object.

Figure~\ref{fig:efake_frate} shows the derived $R_{\Pe}$ factor as a function of \ETg. 
The electron proxy sample is reweighted by $R_{\Pe}$ depending on the \pt\ of the electron object.

%%% this file has charged PF veto included, need to swap out
\begin{figure}[htbp]
  \begin{center}
    \includegraphics[width=0.48\linewidth]{Analysis/Figures/frate_data_ptalt.pdf} 
    \caption{
      Electron to photon fake rate $R_{\Pe}$.
    }
    \label{fig:efake_frate}
  \end{center}
\end{figure}

\section{Misidentified hadrons}
\label{sec:hfake}

The estimation of hadron misidentification background proceeds in multiple steps. 
First, the fraction of hadronic objects within a pool of photon candidate objects in the photon plus jet control region is measured. 
This measurement is described in detail in Section~\ref{subsec:pixelveto}.
Figure~\ref{fig:impurity-compsb} and Table~\ref{tab:hfake-impurity-systs} show the final impurity and associated uncertainties as a function of \pt. 

\begin{table}[htbp]
  \centering
  \begin{tabular}{ |c|c|c c c c| }
\hline
\multicolumn{6}{ |c| }{Impurity (\%) for barrel medium-pixel-monoph photons in data} \\
\hline
$p_{T}$ Range & Nominal & \multicolumn{4}{ |c| }{Sources of Systematic Uncertainty} \\
 (GeV) & & Sideband & CH Iso Shape & Signal Shape & Bgkd. Stats \\
\hline
 (175, 200)  & $4.31 \pm 0.21$ & 0.09 & 0.18 & 0.05 & 0.04 \\
 (200, 250)  & $3.39 \pm 0.17$ & 0.01 & 0.16 & 0.06 & 0.03 \\
 (250, 300)  & $2.44 \pm 0.22$ & 0.14 & 0.16 & 0.06 & 0.05 \\
 (300, 350)  & $1.99 \pm 0.23$ & 0.12 & 0.16 & 0.07 & 0.08 \\
 (350, 400)  & $1.43 \pm 0.28$ & 0.23 & 0.11 & 0.05 & 0.10 \\
 (400, $\infty$)  & $0.63 \pm 0.30$ & 0.27 & 0.09 & 0.05 & 0.05 \\
\hline
\end{tabular}

  \caption{Impurities for photons as a function of \pt.}
  \label{tab:hfake-impurity-systs}
\end{table}

\begin{figure}[htbp]
  \centering
  \includegraphics[width=0.45\linewidth]{Analysis/Figures/hfake/plot_impurity_barrel_medium.pdf}
    \caption{
    Impurities for photons as a function of \pt. 
    The different bands show the effects of adding different stages of the full ID, starting with the baseline ID and isolation and successively adding the pixel seed veto.
  }
  \label{fig:impurity-compsb}
\end{figure}

Following this measurement, another control sample is formed where the photon in the photon plus jet sample is replaced by a hadronic proxy object. 
The hadronic proxy object is a reconstructed photon object which pass the photon ID described in Section~\ref{subsec:objects} with the execption of failing at least one of the following cuts: 
\begin{itemize}
\item $\sieie < 0.01022$
\item PF Charged Hadron isolation $< 0.441$ GeV .
\end{itemize}
Additionally, we apply a $\met < 60$ GeV cut to make this region orthogonal to the signal region of the analysis.
  
The hadronic transfer factor $R_{h}$, which measures the rate at which hadronic proxy objects result in hadrons that are misidentified as candidate photons, is obtained by dividing the estimated number of misidentified hadrons in the photon plus jet sample by the number of events in the hadron proxy + jet control region as a function of \pt. 
Figure~\ref{fig:hadronTFactor} shows the transfer factor $R_{h}$ along with the various distributions used for its derivation.

\begin{figure}[htbp]
  \begin{center}
    \includegraphics[width=0.45\linewidth]{Analysis/Figures/hfake/distributionsNom.pdf}
    \includegraphics[width=0.45\linewidth]{Analysis/Figures/hfake/tfactorNom.pdf}
    \includegraphics[width=0.45\linewidth]{Analysis/Figures/hfake/distributionsTight.pdf}
    \includegraphics[width=0.45\linewidth]{Analysis/Figures/hfake/tfactorTight.pdf}
    \includegraphics[width=0.45\linewidth]{Analysis/Figures/hfake/distributionsLoose.pdf}
    \includegraphics[width=0.45\linewidth]{Analysis/Figures/hfake/tfactorLoose.pdf}
    \caption{
      Left: The \pt\ distribution of the candidate photon object in the photon + jet control sample (black), the result of scaling it with the impurity (red), and the \pt\ distribution of the hadronic proxy object in the proxy + jet control sample (blue).
      Right: Hadronic transfer factor $R_{h}$, which is the ratio of the red and blue distributions in the left plot. 
      Top: Nominal hadron proxy object. 
      Middle: Tighter hadron proxy object. 
      Bottom: Looser hadron proxy object.
    }
    \label{fig:hadronTFactor}
  \end{center}
\end{figure}

Finally, a third control sample of events with a hadronic proxy object and $\met > 170$ GeV is prepared. 
Under the assumption that the $R_{h}$ stays constant regardless of whether the event has a high-\pt\ jet or \met, this proxy plus \met\ sample is then weighted by $R_{h}$ to arrive at an estimate of the misidentified hadron plus \met\ background of this analysis.

%%% need to update these numbers
To estimate the uncertainty on this background, we repeat the above method using tighter and looser definitions of the hadron proxy object.
The tighter definition differs from the nominal by the following cuts:
\begin{itemize}
\item $\rho$-corrected PF Neutral Hadron isolation $< 0.264 + 0.014 \times \pt^\gamma + 0.000019 \times \left( \pt^\gamma \right)^2$. 
\item $\rho$-corrected PF Photon isolation $< 2.362 + 0.0053 \times \pt^\gamma$,
\end{itemize}
and the looser definition differs from the nominal by the following cuts:
\begin{itemize}
\item $\rho$-corrected PF Neutral Hadron isolation $< 10.910 + 0.014 \times \pt^\gamma + 0.000019 \times \left( \pt^\gamma \right)^2$. 
\item $\rho$-corrected PF Photon isolation $< 3.630 + 0.0053 \times \pt^\gamma$.
\end{itemize}

The different distributions from the nominal, tight, and loose selections are shown in Figure~\ref{fig:hadronFakeShapes}. 
The tight and loose shapes are taken as the one sigma band around the nominal estimate. 
Additionally, there is an uncertainty coming from the estimation of the photon purity. 
Figure~\ref{fig:hadronFakePurity} shows the resulting shapes from moving the shapes generated by a one sigma shift in the purity.

\begin{figure}[htbp]
  \begin{center}
    \includegraphics[width=0.48\linewidth]{Analysis/Figures/hfake/shape_sample.pdf}
    \caption{
      The \pt\ distribution of the estimated contribution from hadronic fakes in the signal region. 
      The distribution labeled Up (Down) comes from the tighter (looser) selection. 
      The systematic uncertainty resulting from this variation is around 5\% at the low end of our \pt\ range and increases to 15\% after $\pt > 400$ GeV.
    }
    \label{fig:hadronFakeShapes}
  \end{center}
\end{figure}

\begin{figure}[htbp]
  \begin{center}
    \includegraphics[width=0.48\linewidth]{Analysis/Figures/hfake/shape_purity.pdf}
    \caption{
      The \pt\ distribution of the estimated contribution from hadronic fakes in the signal region. 
      The distribution labeled Up (Down) comes from varying the purity one sigma up (down). 
      The systematic uncertainty resulting from this variation is around 5\% at the low end of the \pt\ range and increases to 20\% after $\pt > 400$ GeV.
    }
    \label{fig:hadronFakePurity}
  \end{center}
\end{figure}

\section{Spikes}
\label{sec:spikes}

\section{Beam halo}
\label{sec:beam_halo}

Figure~\ref{fig:halophi} shows the $\phi$ distribution of the halo showers obtained from the single photon data set, requiring $\ETg > 175\GeV$ and $\met > 170\GeV$. 
Here, halo showers are defined as photon objects that fail the MIP total energy cut in events where beam halo MET filter (one component of the ``MET filters'' mentioned in Section~\ref{subsec:selection}). 
The distribution is shifted and folded to make the peaking behavior clear. 
The resulting variable on the horizontal axis is named $\phi'$.

\begin{figure}[tbp]
  \begin{center}
    \includegraphics[width=0.45\textwidth]{Analysis/Figures/haloPhiFolded.pdf}
    \caption{
      Folded $\phi'$ distribution of the halo sample.
    }
    \label{fig:halophi}
  \end{center}
\end{figure}

The splitting of the signal region can be thought of as a two-bin fit. 
Collision processes occupy the relative fractions of phase space in the horizontal ($H$) and vertical ($V$) signal regions, $C_{H} = 1/\pi$ and $C_{V} = (\pi-1)/\pi$, respectively. 
The corresponding fractions for beam halo events are determined by selecting a halo-enriched sample where the halo identification is inverted. 
Thus, a fit of the two signal regions provides an estimate of the overall normalization of the beam halo background, denoted $h$. 
The \ETg\ dependence of the halo background is encoded in \nhalo[,i], the unit-normalized beam halo prediction in the $i^\mathrm{th}$ bin of the signal region $K \in \{H,V\}$.
Using the notation introduced in Section~\ref{sec:irreducible}, the total estimated background \TK\ in the two signal regions are
\begin{equation}
\begin{aligned}
  \TK[,i] & = C_{K} (\NZg[i] + \NWg[i]) + h \nhalo[,i] + C_{K} b_{K,i} \\
          & = C_{K} (1 + {\fZW[i]}^{-1}) \NZg[i] + h \nhalo[,i] + C_{K} b_{K,i},
\end{aligned}
\end{equation}
where $b_{K,i}$ is the total contribution to bin $i$ of region $K$ from electron and hadron misidentification, ECAL spikes, and other minor SM background processes.

\section{Other minor SM background processes}
\label{sec:minorsm}
The SM \ttg, VV\Pgg, \zllg, $\PW\to\ell\PGn$, and \gj\ processes are minor ($\sim$10\%) background processes in the signal region. 
Although \zllg\ and \gj\ do not involve high-\pt\ invisible particles, the former can exhibit large \met\ when the leptons fail to be reconstructed, and the latter when jet energy is severely mismeasured. 
The estimates for all five processes are taken from \MGvATNLO simulations at LO in QCD and can be found in Tables~\ref{tab:yield_mask_horizontal} and~\ref{tab:yield_mask_vertical}.

\section{Statistical Interpretation}
\label{sec:interpretation}

Free parameters of the fit are the yield of \zinvg\ background in each bin of the signal regions (\NZg[i]) and the overall normalization of the beam halo background ($h$). 
Bin-by-bin yields of \wlng\ and \zllg\ samples in all regions are related to the yield of \zinvg\ through the MC prediction through the transfer factors defined in Section~\ref{sec:irreducible}. 
The transfer factors are allowed to shift within the aforementioned theoretical and experimental uncertainties.

The background-only likelihood that is maximized in the fit is
\begin{equation}
\begin{aligned}
  \mathcal{L} & = \prod_{i} \left\{ \mathcal{L_{\text{signal}}} \times \mathcal{L_{\text{single-lepton}}} \times \mathcal{L_{\text{dilepton}}} \right\} \times \mathcal{L_{\text{nuisances}}} \\
  & = \prod_{i} \left\{
    \prod_{K=H,V} \mathcal{P}\left( d_{K, i} \left| \TK[,i] (\vec{\theta} \right.) \right) \times \prod_{\ell=\Pe,\Pgm} \mathcal{P}\left( d_{\ell\Pgg, i} \left| \Tl[,i] (\vec{\theta}) \right. \right)
    \times \prod_{\ell=\Pe,\Pgm} \mathcal{P}\left( d_{\ell\ell\Pgg, i} \left| \Tll[,i] (\vec{\theta}) \right. \right)
    \right\}  \times \prod_{j} \mathcal{N}(\theta_j) \\
  & = \prod_{i} \left\{
  \begin{gathered}
    \prod\limits_{K=H,V} \mathcal{P}\biggl( d_{K, i} \biggl| \left(1 + {\fZW[,i]}^{-1}(\vec{\theta})\right) C_{K} \NZg[i] + h \nhalo[,i](\vec{\theta}) + C_{K} b_{K, i}(\vec{\theta}) \biggr. \biggr) \\
    \times \prod\limits_{\ell=\Pe,\Pgm} \mathcal{P}\biggl( d_{\ell\Pgg, i} \biggl| \frac{\NZg[i]}{\RWl[,i](\vec{\theta}) \fZW[,i] (\vec{\theta})} + b_{\ell\Pgg, i}(\vec{\theta}) \biggr. \biggr) \\
    \times \prod\limits_{\ell=\Pe,\Pgm} \mathcal{P}\biggl( d_{\ell\ell\Pgg, i} \biggl| \frac{\NZg[i]}{\RZll[,i](\vec{\theta})} + b_{\ell\ell\Pgg, i}(\vec{\theta}) \biggr. \biggr)
  \end{gathered} \right\}
  \times \prod_{j} \mathcal{N}(\theta_j),
\end{aligned}
\end{equation}
following the notation introduced in Section~\ref{sec:irreducible}, and where $\mathcal{P}
(n\vert\lambda)$ is the Poisson probability of $n$ for mean $\lambda$, $\mathcal{N}$ denotes the unit normal distribution, and $d_{X,i}$ is the observed number of events in bin i of region X.
Systematic uncertainties are treated as nuisance parameters in the fit and are represented by
$\vec{\theta}$.
Each quantity $Q_{j}$ with a nominal value $\overline{Q}_{j}$ and a standard deviation of the systematic uncertainty $\sigma_{j}$ appears in the likelihood function as $\overline{Q}_{j}\exp(\sigma_{j}\theta_{j})$.

\section{Results}
\label{sec:results}

\subsection{Pre-fit and post-fit distributions}
\label{subsec:distributions}

\begin{figure}[htbp]
  \centering
    \includegraphics[width=0.45\textwidth]{Analysis/Figures/bonly_diel.pdf}
    \includegraphics[width=0.45\textwidth]{Analysis/Figures/bonly_dimu.pdf}
    \includegraphics[width=0.45\textwidth]{Analysis/Figures/bonly_monoel.pdf}
    \includegraphics[width=0.45\textwidth]{Analysis/Figures/bonly_monomu.pdf}
    \caption{
      Comparison between data and MC simulation in the four control regions: 
      \Pe\Pe\Pgg\ (upper left), 
      \Pgm\Pgm\Pgg\ (upper right), 
      \Pe\Pgg\ (lower left), 
      \Pgm\Pgg\ (lower right) 
      before and after performing the simultaneous fit across all the control samples and signal region, and assuming absence of any signal.
      The last bin of the distribution includes all events with $\ETg > 1000\GeV$. 
      The ratios of data with the pre-fit background prediction (red dashed) and post-fit background prediction (blue solid) are shown in the lower panels. 
      The bands in the lower panels show the post-fit uncertainty after combining all the systematic uncertainties.
}
    \label{fig:postfitCR}
\end{figure}

\begin{figure}[htbp]
  \centering
    \includegraphics[width=0.45\textwidth]{Analysis/Figures/bonly_horizontal.pdf}
    \includegraphics[width=0.45\textwidth]{Analysis/Figures/bonly_vertical.pdf}
    \caption{
      Observed \ETg\ distributions in the horizontal (left) and vertical (right) signal regions compared with the post-fit background expectations for various SM processes.
      The last bin of the distribution includes all events with $\ETg > 1000\GeV$. 
      The expected background distributions are evaluated after performing a combined fit to the data in all the control samples and the signal region. 
      The ratios of data with the pre-fit background prediction (red dashed) and post-fit background prediction (blue solid) are shown in the lower panels. 
      The bands in the lower panels show the post-fit uncertainty after combining all the systematic uncertainties. 
      The expected signal distribution from a 1\TeV vector mediator decaying to 1\GeV DM particles is overlaid.
    }
    \label{fig:postfitSR}
\end{figure}

Figure~\ref{fig:postfitCR} shows the observed \ETg\ distributionsin the four control regions compared with the results from simulations before and after performing the simultaneous fit across all the control samples and signal region, and assuming absence of any signal.
Figure~\ref{fig:postfitSR} shows the observed \ETg\ distributions in the horizontal and vertical signal regions compared with the results from simulations before and after performing a combined fit to the data in all the control samples and the signal region. 
The observed distributions are in agreement with the prediction from SM and noncollision backgrounds.

\begin{table}[htbp]
\centering
\caption{Expected event yields in each \ETg\ bin for various background processes in the horizontal signal region.
         The background yields and the corresponding uncertainties are obtained after performing a combined fit to data in all the control samples, excluding data in the signal region.
         The observed event yields in the horizontal signal region are also reported.}
\label{tab:yield_mask_horizontal}
\begin{tabular}{ lcccccc }
\hline
\rule[-1.2ex]{0pt}{3.8ex}\ETg~[\GeVns{}]      &         [175,  200] &         [200,  250] &         [250,  300] &         [300,  400] &         [400,  600] &         [600, 1000] \\
\hline
$\PZ\Pgg$        & $  81.2 \pm   8.0 $ & $  88.2 \pm   8.4 $ & $  38.8 \pm   4.8 $ & $  26.8 \pm   3.7 $ & $   8.8 \pm   1.9 $ & $   1.4 \pm   0.7 $ \\
$\PW\Pgg$        & $  27.9 \pm   3.7 $ & $  29.9 \pm   3.9 $ & $  11.4 \pm   1.7 $ & $   6.3 \pm   1.2 $ & $   1.4 \pm   0.4 $ & $   0.1 \pm   0.1 $ \\
Misid. electrons & $  22.5 \pm   2.7 $ & $  25.7 \pm   2.7 $ & $  10.5 \pm   1.0 $ & $   8.2 \pm   0.7 $ & $   2.7 \pm   0.2 $ & $   0.5 \pm   0.0 $ \\
Misid. hadrons   & $   5.2 \pm   2.2 $ & $   9.3 \pm   1.8 $ & $   3.1 \pm   0.7 $ & $   1.0 \pm   0.3 $ & $   0.4 \pm   0.1 $ & $   0.0 \pm   0.0 $ \\
Other SM         & $  13.6 \pm   2.0 $ & $  19.6 \pm   1.3 $ & $  13.9 \pm   0.4 $ & $   4.2 \pm   0.2 $ & $   0.8 \pm   0.0 $ & $   0.1 \pm   0.0 $ \\
ECAL spikes      & $   4.3 \pm   1.3 $ & $   2.7 \pm   0.8 $ & $   0.5 \pm   0.1 $ & $   0.1 \pm   0.0 $ & $   0.0 \pm   0.0 $ & $   0.0 \pm   0.0 $ \\
Total prediction & $ 154.6 \pm   8.3 $ & $ 175.4 \pm   8.8 $ & $  78.2 \pm   5.3 $ & $  46.6 \pm   4.0 $ & $  14.1 \pm   2.1 $ & $   2.1 \pm   0.8 $ \\
Observed         & $ 150   \pm  12   $ & $ 166   \pm    13 $ & $  76.0 \pm   8.7 $ & $  44.0 \pm   6.6 $ & $  19.0 \pm   4.4 $ & $   4.0 \pm   2.0 $ \\
\hline
\end{tabular}
\end{table}

\begin{table}[htbp]
\centering
\caption{Expected event yields in each \ETg\ bin for various background processes in the vertical signal region.
         The background yields and the corresponding uncertainties are obtained after performing a combined fit to data in all the control samples, excluding data in the signal regions.
         The observed event yields in the vertical signal region are also reported.}
\label{tab:yield_mask_vertical}
\begin{tabular}{ lcccccc }
\hline
\rule[-1.2ex]{0pt}{3.8ex}\ETg~[\GeVns{}]      &         [175,  200] &         [200,  250] &         [250,  300] &         [300,  400] &         [400,  600] &         [600, 1000] \\
\hline
$\PZ\Pgg$        & $ 172   \pm    17 $ & $ 190   \pm  18   $ & $  83   \pm  10   $ & $  58.6 \pm   7.9 $ & $  18.0 \pm   3.9 $ & $   3.1 \pm   1.6 $ \\
$\PW\Pgg$        & $  59.9 \pm   7.8 $ & $  63.6 \pm   7.8 $ & $  24.6 \pm   3.5 $ & $  13.4 \pm   2.4 $ & $   3.0 \pm   0.8 $ & $   0.3 \pm   0.2 $ \\
Misid. electrons & $  48.4 \pm   5.6 $ & $  56.2 \pm   5.1 $ & $  23.4 \pm   1.8 $ & $  15.7 \pm   1.4 $ & $   5.6 \pm   0.4 $ & $   1.2 \pm   0.1 $ \\
Misid. hadrons   & $  15.1 \pm   4.4 $ & $  14.5 \pm   3.1 $ & $   4.2 \pm   0.8 $ & $   2.3 \pm   0.8 $ & $   0.5 \pm   0.1 $ & $   0.1 \pm   0.1 $ \\
Other SM         & $  33.8 \pm   4.1 $ & $  36.6 \pm   2.7 $ & $  13.6 \pm   0.5 $ & $  17.1 \pm   0.6 $ & $   2.4 \pm   0.1 $ & $   0.8 \pm   0.0 $ \\
ECAL spikes      & $   9.3 \pm   2.8 $ & $   5.7 \pm   1.7 $ & $   0.9 \pm   0.3 $ & $   0.3 \pm   0.1 $ & $   0.0 \pm   0.0 $ & $   0.0 \pm   0.0 $ \\
Total prediction & $ 339   \pm  18  $ & $ 366   \pm  19   $ & $ 150   \pm  11   $ & $ 107.5 \pm   8.7 $ & $  29.6 \pm   4.3 $ & $   5.4 \pm   1.7 $ \\
Observed         & $ 301   \pm  17   $ & $ 342   \pm  19   $ & $ 161 \pm  13   $ & $ 107   \pm  10   $ & $  41.0 \pm   6.4 $ & $  12.0 \pm   3.5 $ \\
\hline
\end{tabular}
\end{table}

The expected yields in each bin of \ETg\ for all backgrounds in the horizontal and vertical signal regions after performing a combined fit to data in all the control samples, excluding data in the signal regions, are given in Tables~\ref{tab:yield_mask_horizontal} and~\ref{tab:yield_mask_vertical}, respectively.
The covariances between the predicted background yields across all the \ETg~bins in the two signal regions are shown in Fig.~\ref{fig:correlation_matrix}.
The expected yields together with the covariances can be used with the simplified likelihood
approach detailed in Ref.~\cite{CMS-NOTE-2017-001} to reinterpret the results for models not studied in this thesis

\begin{figure}[hbtp]
  \centering
  \includegraphics[width=0.9\textwidth]{Analysis/Figures/correlation_matrix.pdf}
  \caption{
    Covariances between the predicted background yields in all the \ETg\ bins of the horizontal and vertical signal regions.
    The bin labels specify which signal region the bin belongs to and what number bin it is for that region.}
  \label{fig:correlation_matrix}
\end{figure}.

\subsection{Limits}
\label{subsec:limits}

\begin{figure}[htbp]
  \centering
   \includegraphics[width=0.48\linewidth]{Analysis/Figures/limits_vector.pdf}
    \includegraphics[width=0.48\linewidth]{Analysis/Figures/limits_axial.pdf}
    \caption{
      The ratio of 95\% \CL\ upper cross section limits to the theoretical cross section ($\mu_{95}$), for DM simplified models with vector (left) and axial-vector (right) mediators, assuming $\gq=0.25$ and $\gDM=1$.
      Expected $\mu_{95} = 1$ contours are overlaid in red. 
      The region under the observed contour is excluded. For DM simplified model parameters in the region below the lower violet dot--dash contour, and also above the corresponding upper contour in the right hand plot, cosmological DM abundance exceeds the density observed by the Planck satellite experiment.
    }
    \label{fig:limits}
\end{figure}

Figure~\ref{fig:limits} shows the 95\% \CL\ upper cross section limits with respect to the corresponding theoretical cross section ($\mu_{95}= \sigma_{95\%}/\sigma_{\text{theory}}$) for the  vector and axial-vector mediator scenarios, in the \mmed--\mdm\ plane. 
The solid black (dashed red) curves are the observed (expected) contours of $\mu_{95} = 1$. 
The $\sigma_{\text{theory}}$ hypothesis is excluded at 95\% \CL\ or above in the region with $\mu_{95} < 1$. 
The uncertainty in the expected upper limit includes the experimental uncertainties. 
For the simplified DM LO models considered, mediator masses up to 950\GeV are excluded for values of \mdm\ less than 1\GeV.
