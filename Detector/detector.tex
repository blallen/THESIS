\chapter{The CMS Detector}
\label{sec:cms}

The Compact Muons Solenoid (CMS) detector is one of two general purpose detectors at the Large Hadron Collider.
The original motivation for the experiment was the discovery of the Higgs boson in final states with photons, electrons, and muons.
Towards this end, the subdetectors were built to specify the following goals
\begin{itemize}
\item Unambiguous charge identification of muons with momenta up to 1\TeV
\item 1\GeV mass resolution on 100\GeV pairs of muons, electrons, and photons
\item Efficient triggering and tagging of $\tau$ lepton and $b$ quark decays
\item Good resolution on dijet masses and missing transverse energy
\item Sufficient time resolution to deal with 40MHz of collisions .
\end{itemize}
The four subdetectors are the silicon pixel and strip trackers, the electromagnetic calorimeter (ECAL), the hadronic calorimeter (HCAL), and the muon chambers.
The first three are within the field volume of a 3.8T superconducting solenoid magnet with an internal diameter of six meters.
The muon chambers are embedded in the return yoke of the magnet.
Additionally, there is an online triggering system to reduce readout from 40MHz to $\mathcal{O}(1)$kHz. 

Figure~\ref{} shows a diagram of the CMS detector.
The CMS detector has a length of 22 m and a diameter of 15 m and a cylindrical geometry with concentric barrel shaped detectors in the central region and disc shaped detectors in the forward region.
The following coordinate system is used when working with the CMS detector:
\begin{itemize}
\item distance $z$ along the beam axis
  \begin{itemize}
  \item $z=0$ at the center of the detector
  \item positive corresponds to counter-clockwise as seen from the sky
  \end{itemize}
\item distance $r$ from the beam axis
\item azimuthal angle $\phi$ in the plane orthogonal to the beam axis
  \item pseudorapidity $\eta = - \ln \tan\sfrac{\theta}{2}$ in terms of the polar angle $\theta$ measured with respects to the positive $z$-axis
\end{itemize}
In addition to these four main coordinates, we define the right-handed cartesian $x$ and $y$ coordinates perpendicular to the beam axis, with the positive $x$-axis pointing from the center of the detector to the center of the LHC ring and the positive $y$-axis pointing upwards. 
The spatial separation of two particles is given by $\Delta R = \sqrt{(\Delta\phi)^2 + (\Delta\eta)^2}$.
The fiducial acceptance of the CMS detector is from $0 \le \phi < 2\pi$ and $-5 \le \eta \le 5$.

An additional useful quantity is the momentum in the transverse plane $\ptvec = p_x \hat x + p_y \hat y$ with magnitude $\pt = \sqrt{p_x^2 + p_y^2}$.
In terms of \pt\ and $\eta$, we have the total momentum $p = \pt \cosh \eta$ and the momentum along the beam pipe $p_z = \pt \sinh \eta$. 
Thus, in our detector coordinate frame, the Lorentz-invariant four-momentum $p_i$ of a given particle $i$ is defined by $p_i = (\pt^i, \eta_i, \phi_i, E_i) = (\pt^i, \eta_i, \phi_i, m_i)$, where the first three components are space-like and the last one is time-like. 


\section{Silicon Pixel and Strip Trackers}
\label{sec:cms_tracker}

The tiny dots and thin strips.

\section{Electromagnetic Calorimeter}

Our PbWO$_{4}$ guys.

\section{Hadronic Calorimeter}

Our big brassy boi.

\section{Muon Chambers}

The red ones.

\section{Online Trigger System}

How we choose events.

\section{Detector Simulation}

Gotta get predictions somehow.

% \section{Detector Issues}

% What went wrong.

