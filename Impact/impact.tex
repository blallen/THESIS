\chapter{Comparison with Other Results}

We're not doing this in a vacuum.

\section{Monophoton}

\section{Monojet / Mono-$Z$}

\section{Direct Detection}

The exclusion contours for the vector mediator model shown in Fig.~\ref{fig:limits} are also translated into the $\sigma_{\text{SI}}$--\mdm\ plane, where $\sigma_{\text{SI}}$ are the spin-independent DM--nucleon scattering cross sections as shown in Fig.~\ref{fig:limits_direct}. 
The translation and presentation of the result follows the prescription given in Ref.~\cite{Boveia:2016mrp}.
In particular, to enable a direct comparison with results from direct detection experiments, these limits are calculated at 90\% \CL~\cite{dmforum}.
When compared to the direct detection experiments, the limits obtained from this search provide stronger constraints for DM masses less than 2\GeV for spin independent models.

\begin{figure}[htbp]
  \centering
    \includegraphics[width=0.48\linewidth]{Impact/Figures/limits_direct.pdf}
    \caption{
      The 90\% \CL\ exclusion limits on the $\chi$--nucleon spin-independent scattering cross sections involving the vector operator as a function of the \mdm.
      Simplified model DM parameters of $\gq=0.25$ and $\gDM=1$ are assumed.
      The region to the upper left of the contour is excluded. 
      On the plots, the median expected 90\% \CL\ curve overlaps the observed 90\% \CL\ curve.
      Also shown are corresponding exclusion contours, where regions above the curves are excluded, from the recent results by the CDMSLite~\cite{Agnese:2015nto}, LUX~\cite{Akerib:2016vxi}, PandaX-II~\cite{Cui:2017}, XENON1T~\cite{Aprile:2018}, and CRESST-II~\cite{Angloher:2015ewa}.
    }
    \label{fig:limits_direct}
\end{figure}

\section{Indirect Detection}

The exclusion contours for the axial-vector mediator model shown in Fig.~\ref{fig:limits} are also translated into the $\sigma_{\text{SD}}$--\mdm\ plane, where $\sigma_{\text{SD}}$ are the spin-dependent DM--nucleon scattering cross sections as shown in Fig.~\ref{fig:limits_indirect}. 
The translation and presentation of the result follows the prescription given in Ref.~\cite{Boveia:2016mrp}.
In particular, to enable a direct comparison with results from indirect detection experiments, these limits are calculated at 90\% \CL~\cite{dmforum}.
When compared to the indirect detection experiments, the limits obtained from this search provide stronger constraints for DM masses less than 200\GeV for spin dependent models.

We show the results in Fig.~\ref{fig:limits_indirect}.

\begin{figure}[htbp]
  \centering
    \includegraphics[width=0.48\linewidth]{Impact/Figures/limits_indirect.pdf}
    \caption{
      The 90\% \CL\ exclusion limits on the $\chi$--nucleon spin-dependent scattering cross sections involving the axial-vector operator as a function of the \mdm.
      Simplified model DM parameters of $\gq=0.25$ and $\gDM=1$ are assumed.
      The region to the upper left of the contour is excluded. 
      On the plots, the median expected 90\% \CL\ curve overlaps the observed 90\% \CL\ curve.
      Also shown are corresponding exclusion contours, where regions above the curves are excluded, from the recent results by the PICO-60~\cite{Amole:2017dex}, IceCube~\cite{Aartsen:2016exj}, PICASSO~\cite{Behnke:2016lsk} and Super-Kamiokande~\cite{Choi:2015ara} Collaborations.
    }
    \label{fig:limits_indirect}
\end{figure}
