\chapter{Introduction}

The Standard Model of particle physics was fully experimentally verified in 2012 with the discovery of the Higgs boson by the CMS and ATLAS collaborations at the Large Hadron Collider~\cite{HiggsPaper}.
The Standard Model completely explains all of our observations of the microscopic wold but fails at the scale of the universe, specifically with regards to gravitational interactions.
Many astrophysical observations of gravitational interactions provide strong evidence for the existence of dark matter, potentially explained by a new elementary particle not included in the Standard Model.
Many experiments test the hypothesis that dark matter has a such particle physics origin including searches for the direct production of dark matter at particle colliders.

In this thesis, we focus on a recent search for dark matter in proton-proton collisions resulting in a single high-\pt\ photon and large missing transverse momentum~\cite{Monophoton}.
First, we review the details of the Standard Model and the astrophysical evidence for dark matter.
Then, we describe the Large Hadron Collider, the Compact Muon Solenoid detector, and the methods used to reconstruct the data from the collisions.
Finally, we describe the analysis in detail and summarize the outlook for the future.
