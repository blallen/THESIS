\chapter{The Standard Model}
\label{sec:sm}

The Standard Model (SM) of particle physics describes the physical properties and dynamics of fermions, the fundamental constituents of matter, and their interactions in the language of a Lorentz-invariant quantum field theory (QFT).
The Standard Model consists of a set of fermion fields, shown in Table~\ref{tab:fermions}, and the local gauge symmetry group that acts on them
\begin{equation}
  G_{\text{SM}} = \suthree \times \sutwo \times \uone,
\end{equation}
which is composed of the subgroups
\begin{align}
  G_{\text{QCD}} & = \suthree \qquad \text{and} \nonumber \\
  G_{\text{EWK}} & = \sutwo \times \uone,
\end{align}
corresponding to the strong and electroweak interactions, respectively.
Each fermion field exists in a unique representation of $G_{\text{SM}}$, also summarized in Table~\ref{tab:fermions}.
The possible represenations of \suthree\ are triplet, conjugate, and singlet, denoted by $\mathbf{3}$, $\mathbf{\bar{3}}$, and $\mathbf{1}$, respectively, while the possible representations of \sutwo\ are doublet and singlet, denoted by $\mathbf{2}$ and $\mathbf{1}$, respectively.
All fermions exist in the singlet representation of \uone, only distinguished by differing values of the weak hypercharge $Y$.
Conversely, all fermions in non-singlet representations of \suthree\ and \sutwo\ have the same interaction strength, a feature known as universality.

\begin{table}[htbp]
\centering
\begin{tabular}{ l|c|c|c|c }
  Name & Symbol & $Y$ & \sutwo\ rep. & \suthree\ rep. \\
  \hline
  Left-handed quark & $q_L$ & \sfrac{1}{6} & \textbf{2} & \textbf{3} \\
  Right-handed up-type quark & $u_R$ & \sfrac{2}{3} & \textbf{1} & \textbf{3} \\
  Right-handed down-type quark & $d_R$ & \sfrac{-1}{3} & \textbf{1} & \textbf{3} \\
  \hline
  Left-handed lepton & $\ell_L$ & \sfrac{-1}{2} & \textbf{2} & \textbf{1} \\
  Right-handed charged lepton & $e_R$ & $-1$ & \textbf{1} & \textbf{1} \\
  Right-handed neutrino & $\nu_R$ & \sfrac{1}{6} & \textbf{1} & \textbf{1} \\
\end{tabular}
\caption{
  The categories of SM fermions and the action of the SM local gauge symmetry group $G_{\text{SM}}$.
  Each category contains three members, one for each generation of the Standard Model.
  A corresponding table exists for the charge conjugated fields representing the anti-fermions.
  The subscripts $L$ and $R$ denote whether the field is left- or right-handed.
}
\label{tab:fermions}
\end{table}

For each category of fermion listed in Table~\ref{tab:fermions}, there exist three generations or copies in the Standard Model, identical except for differing masses.
The lepton electroweak doublets contain the left-handed charged leptons and neutrinos
\begin{equation}
  \ell_{L} = \begin{pmatrix} \nu_e \\ e^-_L \end{pmatrix} ,
              \begin{pmatrix} \nu_\mu \\ \mu^-_L \end{pmatrix} ,
              \begin{pmatrix} \nu_\tau \\ \tau^-_L \end{pmatrix} ,
\end{equation}
and the right-handed lepton singlets contain the right-handed projections of the same leptons and neutrinos.
The quark electroweak doublets contain the left-handed up-type and down-type quarks
\begin{equation}
  q_{L} = \begin{pmatrix} u_L \\ d_L \end{pmatrix} ,
           \begin{pmatrix} c_L \\ s_L \end{pmatrix} ,
           \begin{pmatrix} t_L \\ b_L \end{pmatrix} ,
\end{equation}
and the right-handed quark singlets contain the right-handed projections of the same quarks.
% where $d'$, $s'$, and $b'$ are the electroweak eigenstates of the down-type quarks (further explained in Section~\ref{subsec:ewk}).
Quarks also exist in a strong triplet, which will be denoted with a superscript $c$ as necessary.

\section{Strong Interactions}
\label{sec:qcd}

The strong interactions of quarks and gluons are described by quantum chromodynamics (QCD), with the Lagrangian
\begin{equation}
  \label{eqn:lqcd}
  \mathcal{L}_{\text{QCD}} = i \bar q_f^a \slashed D^{ab} q_f^b + m_f \bar q_f^a q_f^a - \frac{1}{4} G_{\mu\nu}^a G^{a,\mu\nu} + \theta \frac{g_s^2}{72\pi^2} \epsilon_{\mu\nu\rho\sigma} G^{c,\mu\nu} G^{c, \rho\sigma},
\end{equation}
where repeated indices are contracted.
The $q_f^a$ are the quark-field Dirac spinors of flavor $f \in \{u,d,c,s,t,b\}$, color $a \in \{r,g,b\}$ the basis elements of the triplet representation of \suthree, and mass $m_f$.
The first term in Equation~\ref{eqn:lqcd} contains the QCD covariant derivative
\begin{equation}
  D_{\mu}^{ab} = \delta^{ab} \partial_\mu - i g_s \sum_{c} t_c^{ab} G_{c,\mu},
\end{equation}
where $g_s$ is the strong interaction coupling strength with associated coupling constant $\alpha_s = g_s^2/(4\pi)$, the $t_c$ are the eight $3\times3$ Hermitian traceless matrices that serve as the generators of the triplet representation of \suthree, and the $G_c$ are the corresponding eight gluon fields.
The third term in Equation~\ref{eqn:lqcd} contains the gluon field strength tensors
\begin{equation}
  G_{\mu\nu}^a = \partial_\mu G_\nu^a - \partial_\nu G_\mu^a - g_s f^{abc} G_\mu^b G_\nu^C,
\end{equation}
where $f^{abc}$ are the structure constants of \suthree.
The non-Abelian structure of the \suthree\ group allows for 3-gluon and 4-gluon interactions in addition to the quark-antiquark-gluon interactions.

The last term in Equation~\ref{eqn:lqcd} violates CP conservation and produces a non-zero electric dipole moment (EDM) for the neutron.
Experimental limits on the neutron EDM constrain the observed QCD vaccum angle $\theta$ to be smaller than $10^{-10}$.
The Peccei-Quinn theory provides a possible explanation for this contradiction by introducing the hypothetical axion particle $a$ with the following Lagrangian % https://arxiv.org/pdf/hep-ph/0607268.pdf
\begin{equation}
  \label{eqn:laxion}
  \mathcal{L}_a = \frac{1}{2} \partial_\mu a \partial^\mu a + \frac{g_s^2}{72 \pi^2} \frac{a}{f_a} \epsilon_{\mu\nu\rho\sigma} G^{c,\mu\nu} G^{c, \rho\sigma},
\end{equation}
where $f_a$ is axion decay constant that determines its characteristic scale.
The second term in Equation~\ref{eqn:laxion} cancels the last term in Equation~\ref{eqn:lqcd} when the axion field dynamically assumes its vacuum expectation value $\langle a \rangle = - f_a \theta$.
The axion is a potential dark matter candidate.

\section{Hadrons and Renormalization}
\label{sec:hadrons}

Due to higher-order corrections to propagators in a QFT, physical quantities such as coupling constants and masses acquire a scale-dependence, where the value of the quantity changes as a function of the probed energy scale $q^2$.
The process of recovering scale-invariance is called renormalization and ensures that any divergent terms from the higher-order corrections cancel out in the physical values.
Given the value of an arbitrary coupling constant $\alpha$ at some known scale $\mu^2$, the value of $\alpha$ at arbitrary scale $q^2$ is  
\begin{equation}
  \label{eqn:rge}
  \alpha (q^2) = \frac{\alpha(\mu^2)}{1 - \alpha(\mu^2) \left[ \Pi(q^2) - \Pi(\mu^2) \right]},
\end{equation}
where $\Pi(q^2)$ and $\Pi(\mu^2)$ are the self-energy correction of the propagator at scales $q^2$ and $\mu^2$.
While these individual terms are separately divergent, their difference is finite and calculable. 

For values of $q^2$ and $\mu^2$ larger than the QCD confinement scale $\lqcd = 218\MeV$, the difference between the gluon self-energy corrections to one-loop order is given by 
\begin{equation}
  \label{eqn:beta_qcd}
  \Pi_s(q^2) - \Pi_s(\mu^2) \approx - \frac{\bqcd}{4\pi} \ln \left( \dfrac{q^2}{\mu^2} \right)
\end{equation}
where $\bqcd$ depends on the number of quark and gluon loops.
For $N_c$ colors and $N_f$ quark flavors with mass below $\abs q$,
\begin{equation}
  \bqcd = \frac{11 N_c - 2 N_f}{12 \pi}. 
\end{equation}
In the Standard Model, $N_c = 3$ and $N_f \le 6$ regardless of energy, thus \bqcd\ is always positive.
Combining Equations~\ref{eqn:rge} and~\ref{eqn:beta_qcd}, the evolution of $\alpha_s$ is given by
\begin{equation}
  \label{eqn:alpha_s}
  \alpha_s (q^2) = \frac{\alpha_s(\mu^2)}{1 + \bqcd \alpha_s (\mu^2) \ln \left( \dfrac{q^2}{\mu^2} \right)} \approx \frac{1}{\beta \ln \left( \dfrac{q^2}{\lqcd^2} \right)}
\end{equation}
for a sufficiently large energy scale $q^2 \gg \lqcd^2$.
Through electron-positron collisions, the value of $\alpha_s$ at the $Z$-pole has been measured to be $\alpha_s (m_Z^2) = 0.1181 \pm 0.0011$.

From Equation~\ref{eqn:alpha_s}, we see that $\alpha_s$ decreases with increasing $q^2$.
At $\abs q \sim 1\GeV$, the value of $\alpha_s$ is of $\mathcal{O}(1)$ confining quarks and gluons to hadrons in a strongly-bound non-perturbative state.
However, $\abs q \gtrsim 100\GeV$, we have $\alpha_s \approx 0.1$ which is small enough that perturbation theory can be used and quarks can be treated as quasi-free particles.
This property of QCD is known as asymptotic freedom.

Below the confinement scale \lqcd, colored objects are always confined to color singlet states and no objects with non-zero color charge propagate as free particles.
This low-energy non-pertubative phenomenon is known as color confinement.
Thus, free quarks and gluons are not observed in nature, only in colorless bound states called hadrons.
The most common states consist of a quark-antiquark pair or three quarks, called mesons and baryons, respectively.
Rarer pentaquark states have recently been found by the LHCb collaboration.

\section{Electroweak Interactions}
\label{sec:ewk}

The electroweak interactions of fermions are described by $\sutwo \times \uone$ gauge group, with the Lagrangian
\begin{equation}
  \label{eqn:lewk}
  \mathcal{L}_{\text{EWK}} = i \bar \psi_i \slashed D \psi_i - \frac{1}{4} \vec W_{\mu\nu} \cdot \vec W^{\mu\nu} - \frac{1}{4} B_{\mu\nu} B^{\mu\nu}
\end{equation}
where repeated indices are contracted and $\psi \supseteq \{q_L, u_R, d_R, \ell_L, e_R, \nu_R\}$ is the set of SM fermions, and the gauge field tensors are given by
\begin{align}
  B_{\mu\nu} & = \partial_\mu B_\nu - \partial_\nu B_\mu \qquad \text{and} \nonumber \\
  \vec W_{\mu\nu} & = \partial_\mu \vec W_\nu - \partial_\nu \vec W_\mu + g \vec W^\mu \times \vec W^\nu ,
\end{align}
where $\vec W_\mu$ and $B_\mu$ are the gauge fields for \sutwo\ and \uone, respectively, and $g$ is the coupling strength for \sutwo.
The first term in Equation~\ref{eqn:lewk} contains the EWK covariant derivative
\begin{equation}
  D_{\mu} = \partial_\mu - i g \vec T \cdot \vec W_\mu - i g' Y B_\mu,
\end{equation}
where $g'$ is the coupling strength for \uone,
$Y$ is the \uone\ hypercharge of the fermion field,
and $\vec T$ are the generators of the doublet representation of \sutwo.
The generators can be written in terms of the Pauli spin matrices $\vec T = \vec \sigma / 2$ and only have non-zero action on left-handed particles. 
The values of the hypercharge $Y$ shown in Table~\ref{tab:fermions} are chosen such that the physical electric charge of each fermion is given by $Q =  T_3 + Y$.

However, this theory of the electroweak interactions is not sufficient to explain the observed behavior of the weak force.
Equation~\ref{eqn:lewk} contains three massless gauge bosons for the weak charge and one massless gauge boson for hypercharge, but experimentally three massive weak gauge bosons and one massless photon have been observed. 
Introducing explicit mass terms of the form $-m_W^2 W_\mu W^\mu$ to the Lagrangian breaks the \sutwo\ gauge invariance as well as making the theory non-renormalizable. 
Spontaneous breaking of the \sutwo\ gauge invariance provides the mechanism we need to provide mass to the weak gauge bosons while maintaining the underlying symmetries and gauge invariance of Equation~\ref{eqn:lewk}.

\section{Electroweak Symmetry Breaking}
\label{sec:ewsb}

The development of a dynamical photon mass in the Bardeen-Cooper-Schrieffer theory of superconductivity~\cite{} provided the template for spontaneous symmetry breaking in the electroweak sector.
Such spontaneous symmetry breaking occurs when the vacuum is degenerate with none of the possible ground states exhibiting the symmetry of the underlying theory.
However, as a consequence of the Goldstone theorem~\cite{}, massless spin-0 Nambu-Goldstone bosons appear after spontaneous symmetry breaking but no such particles are observed in nature.
Fortunately, Brout, Englert, and Higgs as well as Guralnik, Hagen, and Kibble discovered that when an additional field is used to break a gauge symmetry, the gauge bosons acquire a nonzero mass by absorbing the Nambu-Goldstone bosons. 
Building upon these ideas, Glashow, Weinberg, and Salam developed a theory of electroweak unification~\cite{} that explained the observed massive weak bosons in terms of the massless bosons from Equation~\ref{eqn:lewk}.
Finally, t'Hooft and Veltman proved that this model is renormalizable~\cite{}.
We shall walk through the key points of these developments in the remainder of this section.

The \sutwo\ symmetry is broken by introducing a left-handed complex scalar doublet $\phi$ with $Y_\phi = \sfrac{1}{2}$ to the Lagrangian in the following manner
\begin{equation}
  \label{eqn:ewsb}
  \mathcal{L}_{\text{EWK}} \mapsto \mathcal{L}_{\text{EWK}} + \left|D_\mu \phi \right|^2 + \mu^2 \phi^2 - \lambda \left| \phi \right|^4 .
\end{equation}
We choose to write this complex doublet, known as the complex Higgs field, in terms of four real-valued fields so that
\begin{equation}
  \phi = \frac{1}{\sqrt{2}} \begin{pmatrix} \phi_1 + i \phi_2 \\ \phi_3 + i\phi_4 \end{pmatrix} .
\end{equation}
Fortunately, the two self-interaction terms create a Higgs potential with a degenerate global minimum at the vacuum expectation value (vev)
\begin{equation}
  v \equiv \langle | \phi | \rangle = \sqrt{ \frac{\mu^2}{\lambda}},
\end{equation}
and through gauge rotations we set $\langle \phi_{1,2,4} \rangle = 0$, removing three degrees of freedom and producing the three massless Nambu-Goldstone bosons.
The remaining degree of freedom is the real Higgs field $H$ which expresses small peturbations around the vev in the third component of the complex Higgs field $\phi_3 = v + H$.

The kinetic term in Equation~\ref{eqn:ewsb} couples the complex Higgs field to the EWK gauge bosons as follows at the vev
\begin{equation}
  \left| D_\mu \phi \right|^2 = \frac{v^2}{8} \left[ \left(g W_\mu^1 \right)^2 + \left(g W_\mu^2 \right)^2 + \left(g' B_\mu - g W_\mu^3 \right)^2 \right].
\end{equation}
Diagonalizing this term gives rise to the three massive weak bosons and the massless photon that we observe in nature:
\begin{equation}
  \begin{matrix}[l | l]
    W_\mu^{\pm} \equiv \frac{1}{\sqrt{2}} \left( W_\mu^1 \mp W_\mu^2 \right)
    & m_W = \frac{1}{2} vg \\
    Z_\mu \equiv \cos \thetaw W_\mu^3 - \sin \thetaw B_\mu
    & m_Z = \frac{1}{2} v \sqrt{g^2 + (g')^2} \\
    A_\mu \equiv \sin \thetaw W_\mu^3 + \cos \thetaw B_\mu
    & m_A = 0,
  \end{matrix}
\end{equation}
where $\tan \thetaw = g'/g$ and $\cos \thetaw = m_W / m_Z$.
With this, we rewrite Equation~\ref{eqn:lewk} in terms of the observed electromagnetic (EM), charged weaked (CC), and neutral weak (NC) currents as follows:
\begin{align}
  \label{eqn:lewsb}
  \mathcal{L}_{\text{EWK}} % &= \mathcal{L}_{\text{EM}} + \mathcal{L}_{\text{CC}} + \mathcal{L}_{\text{NC}} \nonumber \\
  = \bar \psi_i \left( i \slashed \partial - e Q \slashed A \right) \psi_i
  & - \frac{g}{2\sqrt{2}} \bar \psi_i \left( T^+ \slashed W^+ + T^- \slashed W^- \right) \psi_i - \frac{1}{2} m_W^2 W_\mu^+ W^{-\mu} \nonumber \\
   & - \frac{g}{2 \cos \thetaw} \bar \psi_i (g_V - g_A \gamma^5) \slashed Z \psi_i - \frac{1}{2} m_Z^2 Z_\mu Z^\mu, 
\end{align}
where $e = g' \cos \thetaw$ is the charge of the electron with associated coupling constant $\alpha = e^2/(4\pi)$, $T^\pm = (T_1 \mp i T_2)/\sqrt{2}$ are the weak isospin raising and lowering operators, and $g_V = T_3$ and $g_A = T_3 - 2 Q \sin^2 \thetaw$ are the vector and axial-vector couplings for the neutral weak current.
We can also expand Equation~\ref{eqn:ewsb} about the vev giving us the following Higgs Lagrangian
\begin{align}
  \mathcal{L}_H = \frac{1}{2} \partial_\mu H \partial^\mu H - \frac{1}{2} m_H^2 H^2
  & + \frac{m_H^2}{2 v} H^3 + \frac{2 m_W^2}{v} W_\mu^+ W^{-\mu} H + \frac{m_Z^2}{v} Z_\mu Z^\mu H \nonumber \\
  & + \frac{m_H^2}{8 v^2} H^4 + \frac{m_W^2}{v^2} W_\mu^+ W^{-\mu} H^2 + \frac{m_Z^2}{2v^2} Z_\mu Z^\mu H^2, 
\end{align}
where $m_H = \mu \sqrt{2} $.
Thus, we see that the real Higgs field $H$ has trilinear and quartic couplings to itself and the weak gauge bosons with coupling strengths proportional to the mass squared of the appropriate boson.

\section{Fermion Masses}
\label{sec:yukawa}

Notice that Equation~\ref{eqn:lewsb} does not contain a Dirac mass term like that found in Equation~\ref{eqn:lqcd}.
This is because the term
\begin{equation}
  m \bar \psi \psi = m \left( \bar \psi_L \psi_R + \bar \psi_R \psi_L \right)
\end{equation}
mixes the left-handed and right-handed fermions leading to a Lagrangian that is no longer invariant under \sutwo.
As the observed fermions are not massless, the Lagrangian given in Equation~\ref{eqn:lewk} is incomplete.
Thankfully, introducing Yukawa couplings between the complex Higgs field $\phi$ and the SM fermion fields provides an economical way to add mass terms for the fermions.

First, we start with the terms for charged leptons,
\begin{equation}
  \label{eqn:lyuk}
  \mathcal{L}^{\text{leptons}}_{\text{Y}} = - \bar \ell_L Y_e \phi e_R - \bar e_R Y_e^\dagger \phi^\dagger \ell_L,
\end{equation}
where $Y_e$ is the Yukawa matrix for the charged leptons.
In general, Yukawa matrices and thus mass matrices are non-diagonal and hence we need to convert from the electroweak eigenstates $f_{L,R}$ to the mass eigenstates $\tilde{f}_{L,R} = U^{f}_{L,R} f_{L,R}$ where $U^{f}_{L,R}$ is a unitary matrix.
With this we rewrite Equation~\ref{eqn:lyuk} in terms of the mass eigenstates
\begin{align}
  \mathcal{L}^{\text{leptons}}_{\text{Y}} & = - \bar{\tilde{\ell}}_{L} U^e_L Y_e \phi U^{e\dagger}_R \tilde{e}_{R} - \bar{\tilde{e}}_{R} U^e_R Y_e^\dagger \phi^\dagger U^{e\dagger}_L \tilde{\ell}_{L} \nonumber \\
  & = - \bar{\tilde{\ell}}_{L} \tilde Y_e \phi \tilde{e}_{R} - \bar{\tilde{e}}_{R} \tilde Y_e^\dagger \phi^\dagger \tilde{\ell}_{L}, 
\end{align}
where $\tilde Y_e = U^e_L Y_e U^{e\dagger}_R$ is the diagonalized Yukawa matrix for the charged leptons.
After electroweak symmetry breaking, these terms become
\begin{align}
  \mathcal{L}^{\text{leptons}}_{\text{Y}} & = - \frac{v + H}{\sqrt{2}} \left( \bar{\tilde{e}}_L \tilde Y_e  \tilde{e}_R + \bar{\tilde{e}}_R \tilde Y_e^\dagger \tilde{e}_L \right) \nonumber \\
  & = - \left(1 + \frac{H}{v}\right) \left( \bar{\tilde{e}}_L \tilde M_e \tilde{e}_R + \bar{\tilde{e}}_R \tilde M_e^\dagger \tilde{e}_L \right) \nonumber \\
  & = - \tilde M_e \bar e e - \frac{\tilde M_e}{v} \bar e e H,
    \label{eqn:lmass}
\end{align}
where $\tilde M_e = v \tilde Y_e / \sqrt{2}$ is the diagonalized mass matrix for the charged leptons and $e$ is the set of massive Dirac spinors for the charged leptons.

From Equation~\ref{eqn:lmass}, we see that the Yukawa couplings between the complex Higgs field $\phi$ and the charged leptons result in a Dirac mass term and a coupling to the real Higgs field $H$ that is proportional to the mass of the charged leptons and the vev.
The same procedure is used to introduce mass terms for the down-type quarks whereas for the neutrinos and up-type quarks we must use the conjugate doublet $\phi_c = - i \sigma_2 \phi^*$ in place of $\phi$ to obtain the same result.

\section{Flavor Mixing}
\label{sec:flavor}

For the charged leptons and up-type quarks, it is possible to define a basis of simultaneous electroweak and mass eigenstates, so in practice $\tilde Y_{e,u} = Y_{e,u}$ as $U^{e,u}_L = U^{e,u}_R = \mathit{\mathbf{I}}$.
However, it is not possible to do this for the neutrinos at the same time as the charged leptons or for the down-type quarks at the same time as the up-type quarks.

In Equation~\ref{eqn:lewsb}, the charged current term involves interactions between the up-type and down-type quarks and is not preserved under the transform $f \rightarrow \tilde f$.
Writing this in terms of the mass eigenstates we have
\begin{align}
  \mathcal{L}_{\text{CC}} % & = - \frac{g}{2\sqrt{2}} \bar Q_L \left( T^+ \slashed W^+ + T^- \slashed W^- \right) Q_L \nonumber \\
  & = - \frac{g}{2\sqrt{2}} \left( \bar u_L T^+ \slashed W^+ d_L + \bar d_L T^- \slashed W^- u_L \right) \nonumber \\
  % & = - \frac{g}{2\sqrt{2}} \left( \bar{\tilde{u}}_L U^{u\dagger}_L T^+ \slashed W^+ U^d_L \tilde{d}_L + \bar{\tilde{d}}_L U^{d\dagger}_L T^- \slashed W^- U^u_L \tilde{u}_L \right) \nonumber \\
  & = - \frac{g}{2\sqrt{2}} \left( \bar{u}_L  T^+ \slashed W^+ \vckm \tilde{d}_L + \bar{\tilde{d}}_L T^- \slashed W^- \vckm^\dagger u_L \right),
\end{align}
where $\vckm = U^{u\dagger}_L U^d_L$ is the Cabibbo-Kaboyshi-Maskawa matrix and $u_L = \tilde{u}_L$ by construction.
The CKM matrix is unitary with four free parameters, the mixing angles between quark generations $\phi_{12}$, $\phi_{23}$, and $\phi_{13}$ as well as a CP-violating phase $\delta$.
In terms of these parameters, the CKM matrix is
\begin{equation}
  \vckm = \begin{pmatrix} c_{12} & s_{12} & 0 \\ - s_{12} & c_{12} & 0 \\ 0 & 0 & 1 \end{pmatrix}
  \times \begin{pmatrix} 1 & 0 & 0  \\ 0 & c_{23} & s_{23} \\ 0 & - s_{23} & c_{23} \end{pmatrix}
  \times \begin{pmatrix} c_{13} & 0 & s_{13} e^{-i\delta} \\ 0 & 1 & 0 \\ -s_{13} e^{i\delta} & 0 & c_{13} \end{pmatrix},
\end{equation}
where $s_{ij} = \sin \phi_{ij}$ and $c_{ij} = \cos \phi_{ij}$.
It has been experimentally determined that the CKM is mostly diagonal with $s_{13} \ll s_{23} \ll s_{12} \ll 1$. 

The equivalent mixing matrix for the neutrinos is the Pontecorvo-Maki-Nakagawa-Sakata matrix \upmns, which  converts from the mass eigenstates $\nu_1$, $\nu_2$, and $\nu_3$ to the electroweak eigenstates $\nu_e$, $\nu_\mu$, $\nu_\tau$.
Unlike the CKM matrix, the PMNS is non-diagonal resulting in stronger mixing in the neutrino sector.
The values of the mixing angles $\theta_{12}$, $\theta_{23}$, and $\theta_{13}$ have been measured in neutrino oscillation experiments while the CP-violating phase $\delta'$ has not yet been directly measured.
From cosmological measurements, % of the large-scale structure of the universe,
it is known that the sum of the neutrino masses is less than one eV. 

\section{Summary}

\begin{table}[htbp]
\centering
\begin{tabular}{ c|l|r }
  Parameter & Description & Best Fit Value \\
  \hline
  $\phi_{12}$ & CKM 12-mixing anlge    & $13.1^\circ$ \\
  $\phi_{23}$ & CKM 23-mixing angle    & $2.4^\circ$ \\
  $\phi_{13}$ & CKM 13-mixing angle    & $0.4^\circ$ \\
  $\delta$    & CKM CP-violating phase & 0.995 \\
  \hline
  $\sin^2 \theta_{12}$ & PMNS 12-mixing anlge    & 0.297 \\
  $\sin^2 \theta_{23}$ & PMNS 23-mixing angle    & 0.437 \\
  $\sin^2 \theta_{13}$ & PMNS 13-mixing angle    & 0.0214 \\
  $\delta'$            & PMNS CP-violating phase & 1.35 \\
  \hline
  $\alpha_s$ & QCD coupling constant & 0.1181 \\
  $\alpha$ & EM coupling constant & 1/137.036 \\
  % $g_s$ & \suthree\ coupling constant    & 1.221 \\
  % $g$   & \sutwo\ coupling constant      & 0.652 \\
  % $g'$  & \uone\ coupling constant       & 0.357 \\
  % $v$   & Higgs vacuum expectation value & 246\GeV \\
  \hline
  $\theta$ & QCD vacuum angle & $< 10^{-10}$ 
\end{tabular}
\caption{ The free parameters of the Standard Model, not including masses.}
\label{tab:sm_params}
\end{table}

The Standard Model has a total of 26 free parameters and 17 physical particles.
The parameters are
the twelve Yukawa couplings for the fermions,
the four parameters of the CKM matrix,
the four parameters of the PMNS matrix,
% the three coupling constants $g_s$, $g$, and $g'$,
% the Higgs vacuum expectation value $v$,
the two coupling constants $\alpha_s$ and $\alpha$,
the masses of the weak gauge bosons $m_W$ and $m_Z$,
the mass of the Higgs boson $m_H$,
and the QCD vacuum angle $\theta$.
The best fit values of the SM parameters, excluding masses, are summarized in Table~\ref{tab:sm_params}.

\begin{table}[htbp]
\centering
\begin{tabular}{ l|c|c|c|c }
  Name & Symbol & Spin & Charge & Mass \\
  \hline
  up quark & $u$ & \sfrac{1}{2} & \sfrac{2}{3} & 2.2\MeV \\
  down quark & $d$ & \sfrac{1}{2} & \sfrac{-1}{3} & 4.7\MeV \\
  charm quark & $c$ & \sfrac{1}{2} & \sfrac{2}{3} & 1.28\GeV \\
  strange quark & $s$ & \sfrac{1}{2} & \sfrac{-1}{3} & 95\MeV \\
  top quark & $t$ & \sfrac{1}{2} & \sfrac{2}{3} & 173\GeV \\
  bottom quark & $b$ & \sfrac{1}{2} & \sfrac{-1}{3} & 4.18\GeV \\
  \hline
  electron neutrino & $\nu_e$ & \sfrac{1}{2} & 0 & - \\
  electron & $e$ & \sfrac{1}{2} & $-1$ & 511\keV \\
  muon neutrino & $\nu_\mu$ & \sfrac{1}{2} & 0 & - \\
  muon & $\mu$ & \sfrac{1}{2} & $-1$ & 105\MeV \\
  tau neutrino & $\nu_\tau$ & \sfrac{1}{2} & 0 & - \\
  tau & $\tau$ & \sfrac{1}{2} & $-1$ & 1.78\GeV \\
  \hline
  gluon & g & 1 & 0 & 0 \\
  photon & $\gamma$ & 1 & 0 & 0 \\
  Z boson & $Z$ & 1 & 0 & 91.2\GeV \\
  W boson & $W^\pm$ & 1 & $\pm 1$ & 80.4\GeV \\
  Higgs boson & $H$ & 0 & 0 & 125 GeV
\end{tabular}
\caption{ The physical particles of the Standard Model.}
\label{tab:sm_particles}
\end{table}

The physical particles are the single-particle states of the various mass eigenfields and their properties are summarized in Table~\ref{tab:sm_particles}.
Each of the fermion fields has a corresponding anti-particle with the electromagnetic and color charges inverted.
Most of these single-particle states have finite lifetimes and decay to lower energy configurations.
The only particles whose decays have not been observed are the photon, the electron, the neutrinos, and the proton (a baryon of flavor content uud).
Additionally, stable bound states of protons and neutrons (a baryon of flavor content udd) exist in the form of atomic nuclei.
