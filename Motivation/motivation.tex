\chapter{Motivation}

Why I did this.

\section{The Standard Model}
\label{sec:sm}

The Standard Model (SM) of particle physics describes the physical properties and dynamics of fermions, the fundamental constituents of matter, and their interactions in the language of a Lorentz-invariant quantum field theory (QFT).
The Standard Model consists of a set of fermion fields, shown in Table~\ref{tab:fermions} and the local gauge symmetry group that acts on them
\begin{equation}
  G_{\text{SM}} = \text{SU}(3)_C \times \text{SU}(2)_L \times \text{U}(1)_Y,
\end{equation}
which is composed of the subgroups
\begin{align}
  G_{\text{QCD}} & = \text{SU}(3)_C \qquad \text{and} \\
  G_{\text{EWK}} & = \text{SU}(2)_L \times \text{U}(1)_Y,
\end{align}
corresponding to the \textit{strong} and \textit{electroweak} interactions, respectively.
Each fermion field exists in a unique representation of $G_{\text{SM}}$, also summarized in Table~\ref{tab:fermions}.
The possible represenations of SU(3)$_C$ are triplet, conjugate, and singlet, denoted by $\mathbf{3}$, $\mathbf{\bar{3}}$, and $\mathbf{1}$, respectively, while the possible representations of SU(2)$_L$ are doublet and singlet, denoted by $\mathbf{2}$ and $\mathbf{1}$, respectively.
All fermions exist in the singlet representation of U(1), only distinguished by differing values of the weak hypercharge $Y$.
Conversely, all fermions in non-singlet representations of SU(3)$_C$ and SU(2)$_L$ have the same interaction strength, a feature known as universality.

\begin{table}[htbp]
\centering
\caption{
  The categories of SM fermions and the action of the SM local gauge symmetry group $G_{\text{SM}}$.
  Each category contains three members, one for each generation of the Standard Model.
  A corresponding table exists for the charge conjugated fields representing the anti-fermions.
  The subscripts $L$ and $R$ denote whether the field is left- or right-handed.
}
\label{tab:fermions}
\begin{tabular}{ l|c|c|c|c }
  Name & Symbol & $Y$ & SU(2) rep. & SU(3) rep. \\
  \hline
  \hline
  Left-handed quark & $q_L$ & $\sfrac{1}{6}$ & \textbf{2} & \textbf{3} \\
  Right-handed up-type quark & $u_R$ & $\sfrac{2}{3}$ & \textbf{1} & \textbf{3} \\
  Right-handed down-type quark & $d_R$ & $\sfrac{-1}{3}$ & \textbf{1} & \textbf{3} \\
  \hline
  Left-handed lepton & $\ell_L$ & $\sfrac{-1}{2}$ & \textbf{2} & \textbf{1} \\
  Right-handed charged lepton & $\ell^{-}_R$ & -1 & \textbf{1} & \textbf{1} \\
  Right-handed neutrino & $\nu_R$ & $\sfrac{1}{6}$ & \textbf{1} & \textbf{1} \\
\end{tabular}
\end{table}

For each category of fermion listed in Table~\ref{tab:fermions}, there exist three generations or copies in the Standard Model, identical except for differing masses.
The lepton electroweak doublets contain the left-handed charged leptons and neutrinos
\begin{equation}
  \ell_{iL} = \begin{pmatrix} \nu_e \\ e^{-}_L \end{pmatrix} ,
              \begin{pmatrix} \nu_\mu \\ \mu^{-}_L \end{pmatrix} ,
              \begin{pmatrix} \nu_\tau \\ \tau^{-}_L \end{pmatrix} ,
\end{equation}
with the generations indexed by $i$ and the right-handed lepton singlets contain the right-handed projections of the same leptons and neutrinos.
The quark electroweak doublets contain the left-handed up-type and down-type quarks
\begin{equation}
  q_{iL} = \begin{pmatrix} u_L \\ d'_L \end{pmatrix} ,
           \begin{pmatrix} c_L \\ s'_L \end{pmatrix} ,
           \begin{pmatrix} t_L \\ b'_L \end{pmatrix} ,
\end{equation}
where $d'$, $s'$, and $b'$ are the electroweak eigenstates of the down-type quarks (further explained in Section~\ref{subsec:ewk}). Quarks also exist in a strong triplet, which will be denoted with a superscript $c$ as necessary.

\subsection{Strong Interactions}
\label{subsec:qcd}

The strong interactions of quarks and gluons are described by quantum chromodynamics (QCD), with the Lagrangian
\begin{equation}
  \label{eqn:lqcd}
  \mathcal{L}_{\text{QCD}} = i \bar q_f^a \slashed D^{ab} q_f^b + m_f \bar q_f^a q_f^a - \frac{1}{4} G_{\mu\nu}^a G^{a,\mu\nu} + \theta \frac{g_s^2}{72\pi^2} \epsilon_{\mu\nu\rho\sigma} G^{c,\mu\nu} G^{c, \rho\sigma},
\end{equation}
where repeated indices are contracted.
The $q_f^a$ are the quark-field Dirac spinors of flavor $f \in \{u,d,c,s,t,b\}$, color $a \in \{r,g,b\}$ (he basis element of the triplet representation), and mass $m_f$.
The first term in Equation~\ref{eqn:lqcd} contains the QCD covariant derivative
\begin{equation}
  D_{\mu}^{ab} = \delta^{ab} \partial_\mu - i g_s \sum_{c} t_c^{ab} G_{c,\mu},
\end{equation}
where $g_s$ is the strong interaction coupling strength, $t_c$ are the eight $3\times3$ Hermitian traceless matrices that serve as the generators of the triplet representation of SU(3), and $G_c$ are the corresponding eight gluon fields.
The third term in Equation~\ref{eqn:lqcd} contains the gluon field strength tensors
\begin{equation}
  G_{\mu\nu}^a = \partial_\mu G_\nu^a - \partial_\nu G_\mu^a - g_s f^{abc} G_\mu^b G_\nu^C,
\end{equation}
where $f^{abc}$ are the structure constants of SU(3).
The non-Abelian structure of the SU(3) group means allows for 3-gluon and 4-gluon interactions in addition to the quark-antiquark-gluon interactions.

The last term in Equation~\ref{eqn:lqcd} violates CP conservation and produces a non-zero electric dipole moment (EDM) for the neutron.
Experimental limits on the neutron EDM constraint the $\theta$ parameter to be smaller than $10^{-10}$.
The Peccei-Quinn theory provides a possible method to force $\theta$ to zero by introducing the hypothetical axion particle. The axion is a potential dark matter candidate and will be discussed further in Section~\ref{sec:dm}.

\subsection{Electroweak Interactions}
\label{subsec:ewk}

\section{Dark Matter}
\label{sec:dm}

\subsection{Astrophysical Evidence}

Galactic Rotation Curves and the Bullet Cluster.

\subsection{Simplified Models for the WIMP Paradigm}

It was the hot thing at the time.
