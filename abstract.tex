% $Log: abstract.tex,v $
% Revision 1.1  93/05/14  14:56:25  starflt
% Initial revision
% 
% Revision 1.1  90/05/04  10:41:01  lwvanels
% Initial revision
% 
%
%% The text of your abstract and nothing else (other than comments) goes here.
%% It will be single-spaced and the rest of the text that is supposed to go on
%% the abstract page will be generated by the abstractpage environment.  This
%% file should be \input (not \include 'd) from cover.tex.

In this thesis, we present a search for dark matter in final states containing a high-\pt\ photon and large missing transverse momentum in proton-proton collisions at $\sqrt{s} = 13$ TeV using data collected by the Compact Muon Solenoid (CMS) experiment at the CERN Large Hadron Collider (LHC) corresponding to an integrated luminosity of 35.9 inverse femtobarns.
The main advances in experimental technique compared to previous searches in this final state are the use of data-driven control regions to constrain the main irreducible backgrounds from \zinvg\ and \wlng\ production and an in-depth study of the unique anomolous detector signatures that result in backgrounds due to non-collision processes.
With these improvements, we have the most robust analysis of this kind presented to date.

No deviations from the predictions of the standard model are observed. 
The results are interpreted in the context of dark matter production and limits on new physics parameters are calculated at 95\% confidence level.
We focus on two simplified dark matter production models where new vector and axial mediators couple a new dark dirac fermion to the Standard Model quarks.
These models are chosen as they cover a large class of WIMP-like dark matter particles that show up in many types of more complete new physics models. 
For the two models considered, the observed (expected) lower limits on the masses of the new mediators are  950\,(1150)\GeV for a dark matter particle of a mass of 1\GeV.
